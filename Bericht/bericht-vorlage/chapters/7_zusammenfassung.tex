\section{Zusammenfassung und Ausblick}

Die vorliegende Arbeit konzentriert sich auf die Analyse einer Gebrauchtwagenplattform als Beispiel für einen spezifischen Zwei- oder Mehrseitenmarkt. \\
Um für die definierte Nutzergruppe, Gebrauchtwagenhändler, ein Werkzeug zu bauen, das es erleichtert diese Plattformen leichter zu navigieren, wurde hier ein Visualisierungstool beschreiben. 
Dieses besteht aus drei Visualisierungen unterschiedlicher Arten, darunter ein Scatterplot, ein Parallelplot und ein Starplot. Sie sind darauf ausgerichtet, die Informationen übersichtlich und aussagekräftig zu präsentieren. \\
Trotz der Fortschritte und Möglichkeiten gibt es auch Einschränkungen, die bei der Interpretation der Ergebnisse berücksichtigt werden sollten. Die Genauigkeit der Analysen hängt stark von der Qualität und Vollständigkeit der zugrunde liegenden Daten ab. Eventuelle Unvollkommenheiten oder Verzerrungen in den Daten können Auswirkungen auf die Aussagekraft der Visualisierungen haben. \\
Obwohl die erstellten Visualisierungen einen bedeutenden Beitrag zur Analyse von Gebrauchtwagenangeboten leisten, gibt es noch ungenutzte Potenziale und Möglichkeiten für zukünftige Weiterentwicklungen. Eine sinnvolle Erweiterung könnte beispielsweise in der Integration von vergleichbaren Daten für andere Marktplattformen für Gebrauchtwagen liegen. 
Darüber hinaus könnten die bestehenden Visualisierungen durch zusätzliche visuelle Elemente erweitert werden, um die Interpretation der Daten weiter zu erleichtern. Eine Möglichkeit wäre beispielsweise die Verwendung von Automarkenlogos als Darstellung für Datenpunkte im Scatterplot. Dadurch könnten Nutzer schnell die relevanten Informationen zu den einzelnen Fahrzeugen erfassen und gleichzeitig die visuelle Verbindung zur jeweiligen Automarke herstellen.\\