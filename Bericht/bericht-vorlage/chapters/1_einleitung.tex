\section{Einleitung}

Internetplattformen sind Digitalfirmen, die zwei oder mehr distinkte Nutzergruppen zusammenbringen. Sie werden zwischen B2B, bei denen zwei Geschäftskunden miteinander interagieren, und B2C, bei denen eine Seite durch einen Endkunden gebildet wird. In jedem Fall profitiert jede Seite durch eine Partizipation der gegenüberliegenden Seite am Netzwerk \cite{MUZELLEC2015139}. Dieser Effekt konnte bereits im späten 19.Jahundert bei der schleppend laufenden Adoption des Telefons beobachtet werden. Ein Telefon war nutzlos, wenn es niemanden gab, den man anrufen konnte. Das Netzwerk war erst wertvoll, als es viele Teilnehmer hatte – ein positiver direkter Netzwerkeffekt. \cite{evans2016matchmakers}
Internetmarktplattformen gewannen zuerst in den 1990er mit Firmen wie e-bay und Craiglist Bedeutung und fanden Anfang des Jahrtausends mit AirBnB, Booking.com und heutigen Entwicklungen wie Maschinensucher, wo ganze Industriemaschinen gehandelt werden können, einen steilen Aufstieg. Heute, 30 Jahre später, interagieren Menschen wie selbstverständlich mit solchen Plattformen. 
Selbst große E-Commerce Händler, wie Amazon, haben schon vor Jahren Marktplatzmechaniken in ihre Webseiten integriert. Diese Integration ist nahtlos, dass viele Nutzer gar nicht wissen, dass auf Amazon ein großer Preiskampf zwischen Drittanbietern ausgebrochen ist. (?????)
Gerade deshalb lohnt es sich einen Blick auf die Mechanismen von diesen Plattformen zu werfen und nachzuvollziehen, wie Verkäufer ihre Angebote bewerben.
Dies soll in dieser Arbeit am Beispiel von einer Gebrauchtwagenplattform geschehen. //
Dabei sollen in dieser Arbeit aber auch Markteffekte von PKWs mit unterschiedlichen Eigenschaften nicht außer Acht gelassen werden. Zu diesem Zweck werden die folgenden Forschungsfragen formulieren:

\begin{itemize}
    \item Wie hängt der Angebotspreis mit den angegebenen Merkmalen eines Gebrauchtwagens zusammen?
    \item Was sind die Werte für ein bestimmtes Fahrzeug? Wo befindet es sich im Vergleich zu den anderen Fahrzeugen? 
    \item Sind Preis- und Merkmalsunterschiede zwischen unterschiedlichen Automarken festzustellen, und wenn ja, wie?
\end{itemize}

\subsection{Anwendungshintergrund}
Zu diesem Zweck soll eine Webseite mit verschiedenen Visualisierungen erstellt werden, die es Nutzern ermöglicht schnell einen Überblick über die auf der Plattform vorhandenen Angebote zu erlangen und diese einzuordnen. \\
Die Daten hierzu sollen von einer echten Marktplattform programmatisch abgerufen werden um ein möglichst reales Bild einer solchen Plattform zu geben. Dies führt allerdings auch zu einem hohen Datenaufbereitungsaufwand.\\
\subsection{Zielgruppen}
Die Zielgruppe dieser Arbeit sollen Gebrauchtwagenverkäufer sein. \\
Die erste Zielgruppe zeichnet sich durch eine hohe Fachkenntnisse aus. Im Markt profitiert diese Gruppe von bestehenden Informationsasymmetrien, die sich durch Marktintransparenzen bilden. Durch Internetmarktplattformen sind diese Marktintransparenzen seltener zu finden. Beispielsweise können Käufer Preise vergleichen und können damit besser abschätzen, wie hoch der reale Marktpreis ist. \\
Daher interessieren professionelle Marktteilnehmer, wie auch trotz dieser technologischen Entwicklungen, eine relevante Marge erzielen können. 
Dieser Informationsbedürfnisse möchte diese Arbeit mithilfe von Visualisierungen zu Fahrzeugmerkmalen und Bewerbungsstilen decken. \\


\subsection{Überblick und Beiträge}
Die Daten werden hierzu in zwei unterschiedlichen Arten und Weisen aufgearbeitet. \\
Für die ersten beiden Visualisierungen – einen Scatterplot und einen Parallelplot – werden die beschafften Daten lediglich bereinigt und auf Ebene der einzelnen anzeigen betrachtet. 
Der Scatterplot soll es dem Nutzer hierbei ermöglichen Attribute miteinander gegenüberzustellen und der Parallelplot alle relevanten Merkmale überblicksartig einzusehen. \\
Für die dritte Visualisierung wurden die Daten auf Ebene der einzelnen Automarken betrachtet. Mittels eines Starplots soll dem Nutzer Aufschluss über Trends auf dieser Ebene gegeben werden. \\
